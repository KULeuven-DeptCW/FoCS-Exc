\documentclass{article}
\usepackage{../brackets,../proofenv,../assignment-en,../importsreferences-en}
\usetikzlibrary{automata,calc,fit,shapes,arrows}

\title{Fundamentals of Computer Science\\Solutions \#1}
\author{prof. B. Demoen\\W. Van Onsem}
\date{Maart 2014}
\begin{document}
\maketitle
\begin{exercise}
Find DFAs for the following languages:
\begin{enumerate}
 \item The set of strings $\accl{4,8,1}^{\star}$ containing the substring $481$.
 \item The set of strings in $\accl{a}^{\star}$ whose length is divisible by either $2$ or $7$.
 \item The set of strings $x\in\accl{0,1}^{\star}$ such that $\fun{\#_0}{x}$ is even and $\fun{\#_1}{x}$ is a multiple of $3$.
 \item The set of strings over the alphabet $a$, $b$ containing at least three occurrences of three consecutive $b$'s, overlapping permitted (e.g., the string $bbbbb$ should be accepted).
\end{enumerate}
\begin{answer}See \figref{exc1}.
\begin{figure}[H]
\centering
\importtikzsubfigure{exc1-dfa1}{}
\importtikzsubfigure{exc1-dfa2}{}
\importtikzsubfigure{exc1-dfa3}{}
\importtikzsubfigure{exc1-dfa4}{}
\caption{Exercise 1}
\figlab{exc1}
\end{figure}
\end{answer}
\end{exercise}
\begin{exercise}
Find NFAs for the following languages:
\begin{enumerate}
 \item $\condset{x\in\accl{0,1}^{\star}}{\mbox{$x$ contains an equal number of occurrences of $01$ and $10$}}$
 \item $\condset{x\in\accl{a,b}^{\star}}{\mbox{$x$ contains an even number of $a$'s or an odd number of $b$'s}}$
\end{enumerate}
\begin{answer}See \figref{exc2}.
\begin{figure}[H]
\centering
\importtikzsubfigure{exc2-dfa1}{}
\importtikzsubfigure{exc2-dfa2}{}
\caption{Exercise 2}
\figlab{exc2}
\end{figure}
\end{answer}
\end{exercise}
\begin{exercise}
Construct a nondeterministic finite automaton and an equivalent deterministic one that accept those sequences over the alphabet $\accl{a,b,c,d}$ such that at least one symbol appears precisely twice in the sequence.
\begin{answer}
See \figref{exc3-dfa1} for the NFA and \figref{exc3-dfa2} for the DFA.
\importtikzfigure{exc3-dfa1}{Exercise 3: NFA}
\end{answer}
\end{exercise}
\begin{exercise}
Describe the following languages using regular expressions ($\Sigma=\accl{a,b}$):
\begin{enumerate}
 \item $\condset{x}{\mbox{$x$ contains an even number of $a$'s}}$.
 \item $\condset{x}{\mbox{$x$ contains an odd number of $b$'s}}$.
 \item $\condset{x}{\mbox{$x$ contains an even number of $a$'s or an odd number of $b$’s}}$.
 \item $\condset{x}{\mbox{$x$ contains an even number of $a$'s and an odd number of $b$’s}}$.
\end{enumerate}
Try to simplify the expressions are much as possible (justify each simplification).
\begin{answer}
The following are simplified regular expressions:
\begin{enumerate}
 \item $b^{\star}\brak{ab^{\star}a}^{\star}b^{\star}$.
 \item $a^{\star}\brak{ba^{\star}b}^{\star}a^{\star}ba^{\star}$.
 \item $b^{\star}\brak{ab^{\star}a}^{\star}b^{\star}|a^{\star}\brak{ba^{\star}b}^{\star}a^{\star}ba^{\star}$.
 \item $\brak{aa|bb|baba|abba|baab}^{\star}\brak{b|aba}\brak{aa|bb|baba|abba|baab}^{\star}$.
\end{enumerate}
\end{answer}
\end{exercise}
\begin{exercise}
Convert the following NFA into a DFA using the subset construction:
\importtikzfigure{exc5-dfa1}{NFA to convert to a DFA.}
Clearly show which subset of $\accl{s,t,u}$ corresponds to each state of the deterministic automaton. Omit inaccessible states.
\begin{answer}See \figref{exc5-dfa2}
\importtikzfigure{exc5-dfa2}{The resulting DFA.}
\end{answer}
\end{exercise}
\begin{exercise}
Minimize the following DFA using the quotient construction:
\importtabulartable{exc6-tbl1}{The original DFA.}
\begin{answer}
We start with the partitioning $\accl{\accl{1,2,5,6,7,8},\accl{3,4}}$. This introduces conflicts for $\tupl{1,2}$, $\tupl{1,5}$, $\tupl{1,8}$. We solve this by introducing a new partition containing the last element of each tuple, thus $\accl{\accl{1,6,7},\accl{2,5,8},\accl{3,4}}$. The new conflict is $\tupl{1,7}$, we can however omit this partition since no other partition points to $7$, all problems are resolved and we obtain \tblref{exc6-tbl2}.
\importtabulartable{exc6-tbl2}{The final DFA.}
\end{answer}
\end{exercise}
\begin{exercise}
Convert the DFA from the previous question into a regular expression (use either original DFA or quotient automaton).
\begin{answer}
The obtained regular expression is $a^{\star}b\brak{a^{\star}\brak{b\brak{a|ba^{\star}b}}^{\star}}^{\star}$
\end{answer}
\end{exercise}
\begin{exercise}
Convert the following regular expressions into DFAs:
\begin{enumerate}
 \item $\brak{000^{\star}\cup 111^{\star}}^{\star}$.
 \item $\brak{01\cup 10}\brak{01\cup 10}\brak{01\cup 10}$.
 \item $\brak{0\cup 1\brak{01^{\star}0}^{\star}1}^{\star}$.
\end{enumerate}
Try to simplify as much as possible.
\begin{answer}See \figref{exc8}.
\begin{figure}[H]
\centering
\importtikzsubfigure{exc8-dfa1}{}
\importtikzsubfigure{exc8-dfa2}{}
\importtikzsubfigure{exc8-dfa3}{}
\caption{Exercise 1}
\figlab{exc8}
\end{figure}
\end{answer}
\end{exercise}
\begin{exercise}
Use the pumping lemma to demonstrate that the following languages are not regular:
\begin{enumerate}
 \item $\condset{a^nb^m}{n = 2m}$.
 \item $\condset{x\in\accl{a,b,c}^{\star}}{\mbox{$x$ is a palindrome; i.e., $x=x^R$}}$.
 \item The set $\mbox{PAREN}$ of balanced parentheses ($\Sigma=\accl{(,)}$).
\end{enumerate}
\begin{answer}
\begin{enumerate}
 \item Say the language is indeed regular, then take $s=a^{2\cdot p}b^p$. We know we should be able to subdivide the string $s$ into three parts $s=xyz$. Since $\abs{xy}\leq p$, we know that $y\in a^+$. We say that y has length $l=\abs{y}>0$, now take the string $s'=xy^2z$. The pumping lemma states that this string should be part of the language but $a^{2\cdot p+l}b^p$ does not satisfy the constraints of the language contradicting the pumping lemma.
 \item Say the language is indeed regular, then take $s=a^pb^pa^p$. We know we should be able to subdivide the string $s$ into three parts $s=xyz$ with $\abs{xy}\leq p$. Therefore $y$ consists only out of $a$'s with $y=a^l$, therefore $xy^2z=a^{p+l}b^pa^p$ will yield a string that is not part of the language.
 \item The set $\mbox{PAREN}$ of balanced parentheses ($\Sigma=\accl{(,)}$). The string $(^p)^p$ is part of the language. The theorem states we can subdivide the string in $s=xyz$, with $\abs{xy}\leq p$, thus $y$ only contains $($'s and $\abs{y}=l>0$. This would imply $xy^2z=(^{p+l})^{p}$ is part of the language as well.
\end{enumerate}
\end{answer}
\end{exercise}
\begin{exercise}
Give a context-free grammar for the following languages:
\begin{enumerate}
 \item $\mbox{PAREN}_2$ of balanced strings of parentheses of two types $\brak{}$ and $\fbrk{}$. How would you prove that your grammar is correct?
 \item The set of non-null strings over $\accl{a,b}$ with equally many $a$'s as $b$'s.
\end{enumerate}
\begin{answer}
\begin{enumerate}
 \item \importgram{exc10i1}
 \item \importgram{exc10i2}
\end{enumerate}
\end{answer}
\end{exercise}
\begin{exercise}
Recall that the reverse of a string $x$, denoted $x^R$, is $x$ written backwards. Formally,
\begin{eqnarray}
\epsilon^R=\epsilon;&\brak{xa}^R=ax^R.
\end{eqnarray}
For language $L\subseteq\Sigma^{\star}$, define
\begin{equation}
L^R=\condset{x^R}{x\in L}.
\end{equation}
\begin{enumerate}
 \item Given $u,v\in\Sigma^{\star}$. Show that $\brak{uv}^R=v^Ru^R$.
 \item Given $L_1,L_2\subseteq\Sigma^{\star}$. Show that $\brak{L_1 L_2}^R=L_2^RL_1^R$.
 \item Show that for any $L\subseteq\Sigma^{\star}$, if $L$ is regular, then so is $L^R$.
\end{enumerate}
\begin{answer}
\begin{enumerate}
 \item Without loss of generality, we can say $u=u_1u_2\ldots u_k$ and $v=v_1v_2\ldots v_l$, since the reverse of a string $s=s_1s_2\ldots s_n$ maps to a string $s^R=s_ns_{n-1}\ldots s_1$ and $w=uv$ with:
 \begin{equation}
  w_i=\acclguard{u_i&\xif i\leq k\\v_{i-k}&\xotherwise}\Rightarrow w_i^R=\acclguard{v_{l-i}&\xif i\leq l\\u_{k-i+l}&\xotherwise}.
 \end{equation}
Which is equivalent to $w^R=v_lv_{l-1}\ldots v_1u_ku_{k-1}\ldots u_1$ and thus by extend $w^R=v^Ru^R$.
 \item $L=L_1L_2$ is defined as $\condset{uv}{u\in L_1,v\in L_2}$, the reverse is defined as $L^R=\condset{v^Ru^R}{u\in L_1,v\in L_2}$, since $L_1^R=\condset{x^R}{x\in L_1}$ and $L_2^R=\condset{x^R}{x\in L_2}$, we can state that $L^R=\condset{vu}{v\in L_2^R,u\in L_1^R}$ by the definition of the concatenation of two languages, we can state that $L^R=L_2^RL_1^R$.
 \item Say a language $L$ is regular, than there exists a regular expression $E$. Based on $E$, we can construct a regular expression $E'$ that decides the reverse language inductively:
 \begin{enumerate}
  \item if $E=\epsilon$, then $E'=\epsilon$;
  \item if $E=a$ with $a\in\Sigma$, then $E'=a$;
  \item if $E=\brak{E_1E_2}$ with $E_1,E_2$ regular expressions, then $E'=\brak{E_2'E_1'}$;
  \item if $E=\brak{E_1|E_2}$ with $E_1,E_2$ regular expressions, then $E'=\brak{E_2'|E_1'}$; and
  \item if $E=E_1^{\star}$ with $E_1$ a regular expression, then $E'=E_1'^{\star}$.
 \end{enumerate}
\end{enumerate}

\end{answer}
\end{exercise}
\importtikzfigure{exc3-dfa2}{Exercise 3: DFA}
\end{document}