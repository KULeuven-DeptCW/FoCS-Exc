\documentclass{article}
\usepackage{../brackets,../proofenv,../assignment-en,../importsreferences-en}
\usetikzlibrary{automata,calc,fit,shapes,arrows}

\title{Fundamentals of Computer Science\\Solutions \#2}
\author{prof. B. Demoen\\W. Van Onsem}
\date{March 2014}
\begin{document}
\maketitle
\begin{exercise}
Give a context-free grammar that generates the language $L=\condset{a^ib^jc^k}{i=j\vee j=k \xwhere i,j,k\geq0}$.
\begin{answer}This can be generated with the following context-free grammar:
\importgram{exc12i1}
\end{answer}
\end{exercise}

\begin{exercise}
Give a pushdown automaton for the following languages:
\begin{enumerate}
 \item $\mbox{PAREN}_2$ of balanced strings of parentheses of two types $()$ and $[]$.
 \item $\condset{a^ib^jc^k}{i,j,k\geq0\wedge \brak{i=j\vee i=k}}$.
 \item The complement of the language ${a^nb^n|n\geq 0}$.
 \item $\condset{w\in{0,1}^{\star}}{w=w^R, \mbox{that is, $w$ is a palindrome.}}$
\end{enumerate}
\end{exercise}

\begin{exercise}
Use the pumping lemma for context free languages to show that the following are not context free:
\begin{enumerate}
 \item $\condset{a^nb^nc^n}{n\geq0}$.
 \item $A=\condset{ww}{w\in\accl{0,1}^\star{}}$. Hint consider $A\cap \fun{L}{a^{\star}b^{\star}a^{\star}b^{\star}}$.
\end{enumerate}
\begin{answer}
The following proves these languages are not context-free:
\begin{enumerate}
 \item 
 \item 
\end{enumerate}
\end{answer}
\end{exercise}

\begin{exercise}
Prove that every regular language is context free, by showing how to convert a regular expression directly into an equivalent context-free grammar.
\begin{answer}
We can prove this by induction on the structure of the regular expression. For every regular expression $E$ we define a non-terminal $N_E$, and define it as:
\begin{enumerate}
 \item If $E=\epsilon$, \importgram{exc15i1}
 \item If $E=a$ with $a\in\Sigma$, \importgram{exc15i2}
 \item If $E=E_1E_2$, \importgram{exc15i3}
 \item If $E=E_1|E_2$, \importgram{exc15i4}
 \item If $E=\brak{E_1}^{\star}$, \importgram{exc15i5}
 \item If $E=\phi$, \importgram{exc15i6} (no transition rule).
\end{enumerate}
\end{answer}
\end{exercise}

\begin{exercise}
Use derivatives to show that $abbb$ is an element of $\brak{a|b}b^{\star}a^{\star}$, and that $abbab$ is not.
\end{exercise}

\begin{exercise}
Use derivatives to generate a DFA for $\brak{a|b}b^{\star}a^{\star}$.
\end{exercise}

\begin{exercise}
Given the following automata.

\importtikzfigure{exc19-dfa1}{The automata of exercise 7.}

\begin{enumerate}
 \item Express the language accepted by the automata as a series of linear equations.
 \item Solve the equations to produce a regular expression describing the accepted language.
\end{enumerate}
\end{exercise}

\begin{exercise}
Give a Turing machine (in full detail) to perform
\begin{enumerate}
 \item addition
 \item multiplication
\end{enumerate}
Assume that the input string is of the form $a^n\#a^m$. In the first case, the result, namely the string on the TM when it halts, should be $a^{n+m}$. In the second case, the result should be $a^{n\cdot m}$.
\paragraph{}
The TM should reject if the input string is not of the correct form.
\end{exercise}
\end{document}