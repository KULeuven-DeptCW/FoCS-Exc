\documentclass{article}
\usepackage{../brackets,../proofenv,../assignment-en,../importsreferences-en}
\usetikzlibrary{automata,calc,fit,shapes,arrows}

\title{Fundamentals of Computer Science\\Solutions \#3\\\url{http://goo.gl/XS14aw}}
\author{prof. B. Demoen\\W. Van Onsem}
\date{April 25, 2014}
\begin{document}
\maketitle
\begin{exercise}
Give Turing machines for the following:
\begin{enumerate}
 \item Decide $L=\condset{w\in\accl{0,1}^{\star}}{\mbox{$L$ does not contain twice as many $0$s as $1$s}}$.
 \item Given input $a^m$, the TM halts with $a^{m^2}$ on its tape.
\end{enumerate}
\begin{answer}
\begin{enumerate}
 \item The Turing machine always converts a $1$ to a $2$ when shifting to the right and two $0$'s to $2$'s when shifting to the left. When only $2$ remains, the machine fails. If when shifting not enough $0$'s or $1$'s are encountered, the machine accepts. The Turing machine is depicted in \figref{exc21-tm1}. A \verb+.tm+ file is available called \verb+ex3-tm1.tm+.
\importtikzfigure{exc21-tm1}{A graphical representation of a Turing machines that decides $L$.}
 \item The Turing machine originates from the Turing machine from the previous exercise session. In this exercise session, the Turing machine multiplied two given numbers. Two approaches are possible: copy the given number and then execute the original Turing machine; or modify the original Turing machine and use ``marker'' characters to denote the current position of the first number. \figref{exc21-tm2} shows the second approach. The repository contains \verb+ex3-tm2.tm+ to run a simulation.
\end{enumerate}
\importtikzfigure{exc21-tm2}{A graphical representation of a Turing machines that computes the square-function.}
\end{answer}
\end{exercise}

\begin{exercise}
Design a Turing machine to compute the function $\funm{max}{x,y}=\mbox{the larger of $x$ and $y$}$.
\begin{answer}
We assume that the Turing machine takes as input $a^x\#a^y$. Each iteration, the Turing machine replaces the first and last $a$ of the tape with a $b$. The first group of characters that contains only $b$'s is removed together with the $#$ and the $b$'s of the other group are converted to $a$'s. In case both groups end up with only $b$'s at the same time, the choice is arbitrary. \figref{exc22-tm} shows a graphical representation of such Turing machine. A file \verb+ex3-tm3.tm+. is available in the git repository as well.
\importtikzfigure{exc22-tm}{A graphical representation of a Turing machines that computes the maximum.}
\end{answer}
\end{exercise}

\begin{exercise}
Consider a Turing machine model that uses a $2$-dimensional tape, corresponding to the upper right quadrant of the plane. The head of such a Turing machine can move to the right, left, up or down.
\paragraph{}
Sketch a proof that such a model does not add extra computing power; that is, the class of languages recognized by such Turing machines is the same as the class recognized by basic Turing machines.
\paragraph{}
Be careful to define what the language of the $2$-d TM is. Discuss how to formalize this model.
\begin{answer}

\end{answer}
\end{exercise}

\begin{exercise}
Is is decidable for TM $M$ whether $\fun{L}{M}=\brak{\fun{L}{M}}^R$, that is, is the language of $M$ equal to its reverse?
\begin{answer}
No, this is a consequence of \emph{Rice's theorem}.
\end{answer}
\end{exercise}

\begin{exercise}
Show that $∼HP=\condset{M \#x}{\mbox{$M$ does not halt on $x$}}$ is not decidable.
\begin{answer}
We proof this by contradiction:
\begin{proof}
Say that there exists a Turing machine $T_H$ for $~HP$. This means that $B$ accepts
if $M$ does not halt on $x$. We now construct a \emph{contradiction}-machine $C$:
\begin{equation}
\fun{C}{\tupl{M}}=\fun{B}{\tupl{M,M}}
\end{equation}
In other words $C$ accepts a given Turing machine $M$, if $M$ does not halts on some representation
of $M$. We now calculate $\fun{C}{\tupl{C}}$. If $\fun{C}{\tupl{C}}=\xtrue$, since however this implies that
$C$ does not halt on $\tupl{C}$, but this would imply by the definition of accepting that
$\fun{C}{\tupl{C}}$ is $\xfalse$. Thus contradiction.
\end{proof}
\end{answer}
\end{exercise}

\begin{exercise}
Determine whether the following problems are decidable or undecidable. Give proof.
\begin{enumerate}
 \item Given a TM $M$ and a string $y$, does $M$ write the symbol $\#$ on its tape on input $y$?
 \item Given a context-free grammar $G$, does $G$ generate all strings except $\epsilon$?
 \item Given a TM $M$ and a string $y$, does $M$ ever write a non-blank symbol on its tape on input $y$?
 \item Given a TM $M$ and a string $y$, does the machine ever attempt to move its head left at any point during the computation on $y$?
\end{enumerate}
\begin{answer}
\begin{enumerate}
 \item \textbf{Undecidable}. Otherwise the acceptance problem would be decidable: take a Turing machine $M$ introduce a new character $@$, all the transitions where $\#$, was involved are replaced with transitions with $@$. Now we introduce a new acceptance state $q_a'$ and introduce an edge from the original acceptance state $q_a$ with for each $a\in\Sigma$ $a/\#/R$. If we call with input our modified Turing machine and $y$, the result would be the acceptance of $y$ on $M$.
 \item \textbf{Undecidable}.
 \item \textbf{Undecidable}.
 \item \textbf{Undecidable}.
\end{enumerate}
\end{answer}
\end{exercise}

\begin{exercise}
Determine whether the following languages are decidable or not. If so, give a Turing machine for deciding it. If not, give a proof. Assume that the input $M$ is a description of a Turing machine.
\begin{enumerate}
 \item $\condset{\tupl{M}}{\mbox{$M$ accepts at least two strings of different length.}}$
 \item $\condset{\tupl{M,w}}{\mbox{$M$ accepts $w$ and rejects $w^R$}}$.
\end{enumerate}
\end{exercise}
\begin{answer}
\begin{enumerate}
 \item \textbf{Undecidable}.
 \item \textbf{Undecidable}.
\end{enumerate}

\end{answer}
\end{document}
