\documentclass{article}
\usepackage{../brackets,../proofenv,../assignment-en,../importsreferences-en}
\usetikzlibrary{automata,calc,fit,shapes,arrows}

\title{Fundamentals of Computer Science\\Solutions \#4\\\url{http://goo.gl/XS14aw}}
\author{prof. B. Demoen\\W. Van Onsem}
\date{May 2014}
\begin{document}
\maketitle

\begin{exercise}
Express the minimum and maximum degrees vertices in the graphs $K_n$ (complete graphs) and $K_{n,m}$ (complete bipartite graphs) in terms of $n$ and $m$.
\begin{answer}
In a complete graph $K_n$ each vertex $v_i$ is connected with each other vertex $v_j$. Thus each vertex is connected with $n-1$ edges, the minimum and maximum degree. In a bipartite graph, we consider two types of vertices $u_i$ for $i=1\ldots n$ and $v_j$ for $j=1\ldots m$. All vertices $u_i$ are connected with all vertices $v_j$, thus the degree is $m$. All vertices $v_j$ are connected with all vertices $u_i$, thus the degree is $n$. The minimum degree is thus $\fun{\min}{m,n}$, the maximum $\fun{\max}{m,n}$.
\end{answer}
\end{exercise}

\begin{exercise}
Specify a minimum value for $n$ in terms of $l$ and $m$ such that $K_{l,m}$ is a subgraph of $K_n$.
\begin{answer}
Since a $K_{l,m}$ graph contains $l+m$ vertices, obviously $n$ must be larger or equal to $m+n$. Since a complete graph contains an edge for each two different vertices, there is no restriction to add more vertices, all the edges in the $K_{l,m}$ graph are part of a $K_{l+m}$ graph.
\end{answer}
\end{exercise}

\begin{exercise}
A regular graph is a graph where each vertex has the same number of neighbors, i.e. every vertex has the same degree. A regular graph with vertices of degree $k$ is called a $k$-regular graph.
\paragraph{}
For every $k$-regular graph is there a $\brak{k+1}$-regular graph that contains the $k$-regular graph as a subgraph? Proof or counterexample.
\begin{answer}
Yes.
\begin{proof}
Without loss of generality, we define an ordering into our graph $G=\tupl{V,E,\phi}$ with $V=\accl{v_1,v_2,\ldots,v_n}$ and $E=\accl{e_1,e_2,\ldots,e_l}$ with $l=n\cdot k/2$. We will construct a graph $G'=\tupl{V',E',\phi'}$. Furthermore we consider two cases, one where the number of vertices $n$ is even, and one where $n$ is odd.
\paragraph{}In the case $n$ is even, we can increase the degree of every vertex by adding $n/2$ additional edges $E'=E\cup\accl{e_{l+1},\ldots,e_{l+n/2}}$ with $\fun{\phi}{e_{l+i}}=\tupl{v_{2\cdot i},v_{2\cdot i+1}}$ for $i=1\ldots n/2$. The set of original vertices $V'=V$ is maintained and of course the old edges are not modified $\fun{\phi'}{e_i}=\fun{\phi}{e_i}$ for $i=1\ldots l$. Since the new edges range over the vertices and each vertex occurs only once, the new edges add a degree of one to the vertices. Since the original vertices and edges are maintained, the original graph is clearly a subgraph.
\paragraph{}This does not hold if $n$ is odd, since this implies there is a single vertex (here called $v_n$) that has no neighbor with which an edge can be formed. In that case, we duplicate the graph. More formally $V'=V\cup V_2$ with $V_2=\condset{v'_i}{i\in\accl{1,2,\ldots,n}}$. Furthermore the original edges are maintained and duplicate edges are introduced $E_2=\condset{e'_i}{i\in\accl{1,2,\ldots,l}}$ with $\fun{\phi}{e'_i}=\tupl{v'_j,v'_k}$ given $\fun{\phi}{e_i}=\tupl{v_j,v_k}$. By using the same construction as in the even case, we can add $\floor{n/2}$ additional edges to both parts of the graph. Finally the two vertices $v'_n$ and $v_n$ that have no neighbor are coupled increasing their degree with one as well, or more formally:
\end{proof}
\end{answer}
\end{exercise}

\begin{exercise}
Is there a simple graph on $n$ vertices such that all vertices have distinct degrees? Is there a (general) graph with this property?
\begin{answer}
Not for simple graphs, but for general graphs this holds.
\begin{quote}\begin{proof}
Take $n=2$. In that case there are two vertices $V=\accl{v_1,v_2}$. There are only two possible graphs, one with an edge $e_1=\tupl{v_1,v_2}$ and one without. In both cases, both edges have the same degree. Thus counterexample.
\end{proof}\end{quote}
\begin{construction}[Distinct degree graph]
Given the number of vertices $n$, we first define an ordering on the vertices $V=\accl{v_1,v_2,\ldots,v_n}$. We then introduce edges $E=\accl{e_1,e_2,\ldots,e_m}$ with $m=n\cdot\brak{n-1}/2$. Such that:
\begin{equation}
\fun{\phi}{e_i}=\tupl{v_j,v_j}\mbox{ with }\displaystyle\sum_{k=1}^{j-1}{k}\leq j<\displaystyle\sum_{k=1}^{j}{k}
\end{equation}
In other words, $v_i$ contains $i$ loops.
\end{construction}
\end{answer}
\end{exercise}

\begin{exercise}
Is it possible to draw a graph that has a trail of length seven but no paths of length seven? If so, draw such an example. Is it possible to draw a graph that has a path of length seven but no trail of length seven? If so, draw such an example.
\begin{answer}
\importtikzfigure{exc32-trail}{A graph with a trail of length seven but no path of length seven.}
\importtikzfigure{exc32-path}{A graph with a trail of length seven but no path of length seven.}
\end{answer}
\end{exercise}

\begin{exercise}
Show that for $n\in\NNN$, the graph $K_{n,n}$ has a subgraph isomorphic to $C_k$, a cycle of length $k$, for even $k\in\accl{4,\ldots,2\cdot n}$.
\begin{answer}
First we number the vertices in the bipartite graph $\accl{v_{11},v_{12},\ldots,v_{1n},\ldots,v_{21},v_{22},\ldots,v_{2n}}$ where $v_{1i}$'s belong to the first partition and $v_{2i}$'s to the second partition. A cycle of length $k$ can be defined as follows: $v_{11}\rightarrow v_{21}\rightarrow v_{21}\rightarrow v_{22}\rightarrow\ldots\rightarrow v_{l1}\rightarrow v_{l2}\rightarrow v_{11}$ with $l=k/2$. This is possible because there is an edge between every $v_{1i}$ and $v_{2j}$ and no vertex occurs in the path twice.
\end{answer}
\end{exercise}

\begin{exercise}
Prove that if a graph $G$ has exactly two vertices $u$ and $v$ of odd degree, then $G$ contains a path from $u$ to $v$.
\begin{answer}
\begin{proof}
We first define two new functions: $\fun{l_G}{v}$ the number of loops of a vertex $v$ and $\fun{x_G}{v}$ the number of edges to another vertex of $v$. By definition the degree of a vertex is equal to $\fun{d_G}{v}=2\cdot\fun{l_G}{v}+\fun{x_G}{v}$. Since the degrees of both vertices are odd, we know that $\fun{x_G}{u}$ and $\fun{x_G}{v}$ are odd as well. Since the smallest odd integer is $1$, we know there is at least one edge between $u$ and $v$. The path is simply such edge.
\end{proof}
\end{answer}
\end{exercise}

\begin{exercise}
Define a relation on graphs such that $G$ and $G'$ are related if and only if the maximum degree of $G$ is less than or equivalent to the maximum degree of $G'$. Does the relation satisfy the reflexive, transitive, and antisymmetric properties? Is it a partial order? Does it satisfy the symmetric property? Is the relation an equivalence relation?
\footnote{\begin{description}
 \item[reflexivity] a relation $R\subseteq A\times A$ is reflexive if and only if for all $a\in A$ we have $\tupl{a,a}\in R$.
 \item[transitivity] a relation $R\subseteq A\times A$ is transitive if and only if for all $a,b,c\in A$ if $\tupl{a,b}\in R$ and $\tupl{b,c}\in R$ then $\tupl{a,c}\in R$.
 \item[antisymmetry] a relation $R\subseteq A\times A$ is antisymmetric if and only if for all $a,b\in A$ if $\tupl{a,b}\in R$ and $\tupl{b,a}\in R$, then $a=b$.
 \item[partial order] a relation is a partial order if and only if it is reflexive, transitive and antisymmetric.
 \item[symmetry] a relation $R\subseteq A\times A$ is symmetric if and only if for all $a,b\in A$ if $\tupl{a,b}\in R$ then $\tupl{b,a}\in R$.
 \item[equivalence relation] a relation is equivalence relation if and only if it is reflexive, symmetric and transitive.
\end{description}}
\begin{answer}
For convenience, we will denote this relation with $\preceq$. It is clear that the relation is \textbf{symmetrical}: the maximum degree of two equivalent graphs is the equivalent as well, thus $G\preceq G$ is equivalent with $\funm{maxdeg}{G}\leq\funm{maxdeg}{G}$, since this is a deterministic number, the relation holds.
\paragraph{}The relation is \textbf{transitive} as well, since the relation on integers is transitive as well.
\paragraph{}The relation is \textbf{not antisymmetric}; although the relation on integers is antisymmetric, there are several graphs that map to the same maximum degree. Two graphs with the same maximum degree but different (for instance they have a different number of vertices), map to the same maximum degree thus $G\preceq G'$ and $G'\preceq G$, but $G\neq G'$.
\paragraph{}Since the relation is not antisymmetry, the relation is not a \textbf{partial order};
\paragraph{}The relation is \textbf{not symmetrical}: given two graphs $G$  and $G'$ where $G$ has a smaller maximum degree, $G\preceq G'$ holds, but $G'\preceq G$ doesn't.
\paragraph{}It is \textbf{not an equivalence relation} since the symmetry property does not hold.
\end{answer}
\end{exercise}

\begin{exercise}
It can be shown that there are exactly $11$ trees on seven vertices. Draw these eleven trees, making sure that no two are isomorphic.
\begin{answer}

\end{answer}
\end{exercise}

\begin{exercise}
Let $G$ be a $k$-regular bipartite graph with $k\geq2$. Show that $G$ has no cut edge (or bridge).
\begin{answer}

\end{answer}
\end{exercise}

\begin{exercise}
Show that if a digraph $G$ contains a directed circuit of positive length, then $G$ must contain a cycle.
\begin{answer}

\end{answer}
\end{exercise}

\begin{exercise}
For $k\in\NNN$, let $G$ be a connected graph that contains $2\cdot k$ vertices of odd degree. Show that there exist $k$ edge-disjoint subgraphs $G_1,\ldots,G_k$, such that
\begin{enumerate}
 \item $\fun{E}{G}=\fun{E}{G_1}\cup\fun{E}{G_2}\cup\ldots\cup\fun{E}{G_k}$, and
 \item each $G_i$ has an Eulerian trail.
\end{enumerate}
\begin{hint}
Consider the graph when you add k edges to form an Eulerian graph.
\end{hint}
\begin{note}
Note that two subgraphs $G_1$ and $G_2$ of $G$ are edge-disjoint whenever $\fun{E}{G_1}\cap\fun{E}{G_2}=\emptyset$.
\begin{answer}

\end{answer}
\end{note}
\end{exercise}

\begin{exercise}
Let $G$ be a $k$-regular bipartite graph with $k\geq2$. Show that $G$ has no cut edge (or bridge).
\begin{answer}

\end{answer}
\end{exercise}

\begin{exercise}
For $n\geq2$ show that the complete bipartite graph $K_{n,n}$ is a Hamiltonian graph.
\begin{answer}

\end{answer}
\end{exercise}

\begin{exercise}
Determine all $m,n\in\NNN$ such that the complete bipartite graph $K_{m,n}$ is Hamiltonian.
\begin{answer}

\end{answer}
\end{exercise}

\begin{exercise}
Let $G$ be a connected graph, let $T_1$ and $T_2$ be (the edge sets of) two spanning trees of $G$, and let $e\in T_1\setminus T_2$. Show that:
\begin{enumerate}
 \item there exists $f\in T_2\setminus T_1$ such that $\brak{T_1\setminus\accl{e}}\cup\accl{f}$ is a spanning tree of $G$, and
 \item there exists $f\in T_2\setminus T_1$ such that $\brak{T_2\setminus\accl{f}}\cup\accl{e}$ is a spanning tree of $G$.
\end{enumerate}
\begin{note}
Each of these two properties is called the Tree Exchange Property.
\end{note}
\begin{answer}

\end{answer}
\end{exercise}

\begin{exercise}
A graph in which every vertex has even degree is called an even graph. The complement $E\setminus T$ of a spanning tree $T$ is called a cotree.
\paragraph{}
Show that every cotree of a connected graph $G$ is contained in a unique even subgraph of the graph.
\begin{answer}

\end{answer}
\end{exercise}
\end{document}
