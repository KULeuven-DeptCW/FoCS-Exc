\documentclass{article}
\usepackage{../brackets,../proofenv,../assignment-en,../importsreferences-en}
\usetikzlibrary{automata,calc,fit,shapes,arrows}

\title{Fundamentals of Computer Science\\Solutions \#5\\\url{http://goo.gl/XS14aw}}
\author{prof. B. Demoen\\W. Van Onsem}
\date{May 2014}
\begin{document}
\maketitle
\begin{exercise}
A permutation on a set $\accl{1,\ldots,k}$ is a one-to-one, onto function on this set. When $p$ is a permutation, $p^t$ means the composition of $p$ with itself $t$ times. Let
\begin{equation}
\mbox{PERM-POWER}=\condset{\tupl{p, q, t}}{p = q^t\mbox{ where $p$ and $q$ are permutations on $\accl{1,\ldots,k}$ and $t$ is a binary integer}}
\end{equation}
Show that $\mbox{PERM-POWER}\in P$.
\begin{note}
Note that the most obvious algorithm does not run within polynomial time.
\end{note}
\begin{hint}
First try it where $t$ is a power of $2$.
\end{hint}
\end{exercise}

\begin{exercise}
Show that NP is closed under the Kleene star operation. That is, if $L\in NP$, then $L^{\star}\in NP$.
\end{exercise}

\begin{exercise}
Let $\mbox{DOUBLE-SAT}=\condset{\tupl{\phi}}{\phi\mbox{ has at least two satisfying assignments}}$. Show that $\mbox{DOUBLE-SAT}\in\mbox{NP-complete}$.
\end{exercise}

\begin{exercise}
A colouring of a graph is an assignment of colours to its vertices so that no two adjacent vertices are assigned the same colour. Let
\begin{equation}
\mbox{3COLOUR}=\condset{\tupl{G}}{\mbox{the vertices of $G$ can be coloured with three colours such that no two vertices joined by an edge have the same colour.}}
\end{equation}
Show that $\mbox{3COLOUR}$ is $\mbox{NP-complete}$.
\begin{hint}
The reduction will be of the form $\mbox{3SAT}\leq_p\mbox{3COLOUR}$. Use the subgraphs depicted on \figref{exc48-graph1}.
\importtikzfigure{exc48-graph1}{Subgraphs used to reduce.}
\paragraph{}
The three colours correspond to true ($T$), false ($F$), and a third one (gray) to help enforce that other nodes are coloured $T$ or $F$. Each variable is modeled using the two nodes marked $w$ and $\neg w$. The first graph is used to enforce that $w$ and $\neg w$ are assigned values $T$ and $F$ or $F$ and $T$. The second graph is used to denote binary logical OR between the two legs $l$ and $r$, which may be either variables or other the roots ($\vee$) of other logical ORs. To ensure that an OR is coloured T, every root $\vee$-node is connected to $F$ and $N$.
\begin{example}
For example, the graph on \figref{exc48-graph2} corresponds to the encoding of $w\vee\neg v$.
\importtikzfigure{exc48-graph2}{Example for $w\vee\neg v$.}
\end{example}
\end{hint}
\end{exercise}

\begin{exercise}
Let $\mbox{EQ}_{\mbox{\small{REX}}}=\condset{\tupl{R,S}}{\mbox{$R$ and $S$ are equivalent regular expressions}}$. Show that $\mbox{EQ}_{\mbox{\small{REX}}}\in\mbox{PSPACE}$.
\end{exercise}
\end{document}
