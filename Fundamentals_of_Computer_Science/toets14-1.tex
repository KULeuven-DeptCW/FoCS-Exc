
\documentclass{article}
\title{FCW: gekwoteerde opgave 1\\\url{http://goo.gl/XS14aw}}
\author{Prof. B. Demoen (\url{Bart.Demoen@cs.kuleuven.be})\\ W. Van Onsem (\url{Willem.VanOnsem@cs.kuleuven.be})}
\date{28 maart 2014}
\usepackage{tikz,../assignment-nl,../importsreferences-nl}
\usetikzlibrary{shapes}

\newcommand{\bchr}[2]{\fbrk{\begin{array}{c}#1\\#2\end{array}}}

\newcommand{\cdfa}[2]{\begin{tikzpicture}[statesc/.style={state,scale=0.75}]
\def\cnt{#1}
\def\ra{1.5}
\def\alpha{-360/\cnt}
\node[statesc,initial,accepting] (Q1) at (180:\ra cm) {$q_0$};
\foreach \i in {2,3,...,\cnt} {
  \node[statesc] (Q\i) at (\alpha*\i+180-\alpha:\ra cm) {$q_{\i}$};
}
\foreach \i/\j/\t in {#2} {
  \path[->] (Q\i) edge node[sloped,above]{$\t$} (Q\j);
}
\end{tikzpicture}}
\begin{document}
\maketitle
\richtlijnen{}
\aboutanswers{}
\begin{question}[] We defini\"{e}ren een speciale familie van DFAs
$C_n$ met $n$ toestanden, voor elke $n$. De toestanden zijn geordend
$q_1,q_2,\ldots,q_n$. $q_1$ is de initi\"ele en enige accepterende
toestand. Bovendien kan de DFA enkel overgangen bevatten van een
toestand $q_i$ naar een toestand $q_j$ indien $i<j$. De enige
uitzondering is $q_n$ vanwaar \'{e}\'{e}n boog (met daarop alle
karakters) vertrekt naar $q_1$. Meer formeel:

\vspace{-0.5cm}
\begin{quote}\begin{definition}[Cyclische DFA]
Een \emph{cyclische DFA} met $n \in \NNN$ toestanden is een DFA
$\tupl{Q,\Sigma,\delta,F,q_1}$ met $Q=\accl{q_1,q_2,\ldots,q_n}$,
$\Sigma$ een eindig alfabet, $F=\accl{q_1}$ en $\delta$ voldoet aan:
\begin{eqnarray}
\forall i,j \in [1..n], \forall a \in \Sigma: \delta(q_i,a) = q_j \Rightarrow (j>i) \vee ((j = 1) \wedge (i = n))\\
\forall a\in\Sigma:\fun{\delta}{q_n,a}=q_1
\end{eqnarray}
\end{definition}\end{quote}
\begin{example}
\figref{mrt-q1} toont enkele voorbeelden van een dergelijke DFA.
\begin{figure}[hbt]
\centering
% \importtikzsubfigure{mrt-dfa1}{}
% \importtikzsubfigure{mrt-dfa2}{}
\importtikzsubfigure{mrt-dfa3}{}
\importtikzsubfigure{mrt-dfa4}{}
% \importtikzsubfigure{mrt-dfa5}{}
\caption{Voorbeelden van de familie van cyclische DFAs}
\figlab{mrt-q1}
\end{figure}
\end{example}
\paragraph{}

De verzameling van talen die beslist kunnen worden met een cyclische
DFA noteren we door $RegLan_{C}$ en het is duidelijk dat $RegLan_{C}$
een deelverzameling is van de reguliere talen. Is het een strikte
deelverzameling? Bewijs je antwoord in detail.

\begin{answer}
$\mbox{RegLan}_C$ is een \textbf{strikte deelverzameling} van $\mbox{RegLan}$. We dienen hiervoor te bewijzen dat minstens \'e\'en reguliere taal $L$ niet behoort tot $\mbox{RegLan}_C$. Er zijn verschillende families van talen die we niet kunnen voorstellen met een cyclische DFA. Hieronder een niet-exhaustieve opsomming:
\begin{enumerate}
 \item De lege taal (vermits de start-toestand accepterend is).
 \item Elke taal die geen $\epsilon$ bevat (vermits de start-toestand accepterend is).
 \item Elke eindige taal (vermits een DFA geen $\epsilon$-bogen bevat en een cyclus van minstens lengte $1$ impliceert dat we strings willekeurig dijkwijls kunnen herhalen).
 \item Talen die na een bepaald patroon geen eerder geziene karakters meer bevatten, bijvoorbeeld $L=a^{\star}bc^{\star}$.
\end{enumerate}
\end{answer}
\end{question}

\begin{question}[Minimalisatie van een DFA]
Betreft het minimalisatiealgoritme:
\begin{itemize}

\item[]
Een DFA verdeelt de strings in twee categorieen, accept en reject, en
er zijn dus maar twee soorten toestanden: de accepterende en de
verwerpende toestanden, of $F$ en $Q\setminus F$. Een nieuw soort
$DFA_n$ verdeelt de strings in $n$ categorieen en zijn formele
definitie vertrekt van het tuple
$\tupl{Q,\Sigma,\delta,F_1,F_2,...,F_n,q}$ waarbij $n \geq 2$ en
waarbij de $F_i$ disjunct zijn en hun unie gelijk is aan $Q$. We
veronderstellen dat $\delta$ totaal is. Als een string $s$ {\em eindigt}
in $F_i$ zeggen we dat $s$ behoort tot categorie $i$.

Het minimalisatiealgoritme in de cursus werkt voor een DFA. Geef de
aanpassingen nodig om het te doen werken voor een $DFA_n$, t.t.z. je
moet een minimale $DFA_n$ bekomen die de strings op dezelfde manier
categoriseert. Mogelijk moet je definities aanpassen. Argumenteer
de correctheid.

\end{itemize}
\begin{answer}
Hiervoor dienen we de equivalentie-relatie $\approx$ te veralgemenen:
\begin{equation}
\forall p,q\in Q:p\approx q\Leftrightarrow\forall s\in\Sigma^{\star},\forall i:\brak{\fun{\hat{\delta}}{p,s}\in F_i\leftrightarrow \fun{\hat{\delta}}{q,s}\in F_i}
\end{equation}
Dit leidt tot volgende aanpassing in het minimalisatie-algoritme: in de tweede stap markeren we $\tupl{p,q}$ indien $p$ en $q$ tot een verschillende $F_i$ behoren of formeler:
\begin{quote}
2. Mark $\tupl{p,q}$ if $p\in F_i$,$q\in F_j$ and $F_i\neq F_j$.
\end{quote}
\end{answer}
\end{question}

\begin{question}[Reguliere talen]
Geef voor onderstaande talen een reguliere expressie, een DFA, of
bewijs aan de hand van het pompend lemma dat de taal niet regulier is:
\begin{enumerate}
 \item $L=\accl{\begin{array}{l|l}w\in\accl{a,b}^{\star}&\mbox{voor
elke prefix van $w$ is het absolute verschil tussen}\\&\mbox{het
aantal $a$'s en het aantal $b$'s 0, 1 of 2}\end{array}}$

Een prefix van $w$ is elke string $s$ zodat $w = sv$ waarbij $v \in \Sigma^*$
 \item $L=\condset{a^nb^mc^n}{n,m\in\NNN:n,m\geq 0}$
\end{enumerate}
\begin{answer}
\begin{enumerate}
 \item Deze taal is \textbf{regulier}. Op \figref{mrt-q3-dfa} toont een (minimale) DFA die deze taal beslist. Het idee is om telkens het verschil in $a$'s en $b$'s bij te houden. Indien dit op een willekeurig moment groter wordt dan $2$, komt de DFA in een verwerp-toestand terecht. Indien de DFA op het einde niet op de verwerp-toestand staat, wordt de string geaccepteerd.
 \importtikzfigure{mrt-q3-dfa}{Een DFA die $L_1$ beslist.}
 \item Deze taal is \textbf{niet regulier}. We kunnen eerst de doorsnede nemen met de taal $L=a^{\star}c^{\star}$, in dat geval is $L'=\condset{a^nb^mc^n}{n,m\in\NNN:n,m\geq 0}\cap\accl{a^{\star}c^{\star}}=\condset{a^nc^n}{n\in\NNN}$, een schoolvoorbeeld van een niet-reguliere taal waarvan het pompend lemma in de cursustekst staat uitgewerkt.\\
 Het pompend lemma rechtstreeks gebruiken is minder evident. We werken het hier toch uit. Stel een string $s\in\Sigma^{\star}$ langer dan de pomplengte $p$ met $s=a^pb^pc^p$. Dan moet er een opdeling $s=xyz$ bestaan met $\abs{xy}\leq p$ en $\abs{y}>0$ zodat $\forall i\in\NNN:xy^iz\in L$. Het is evident dat $y$ uit slechts \'e\'en soort karakters kan bestaan: anders pompen we ook strings die meer dan twee keer van karakter verwisselen. Iets wat niet kan volgens de taal. Vermits $\abs{xy}\leq p$, weten we dat $y$ uitsluitend uit $a$'s kan bestaan. Stel $y=a^l$ met $l$ onbekend maar groter dan $0$, dan pompen we voor $i=2$ de string $a^{p+l}b^pc^p$, deze string behoort niet tot de taal bij om het even welke $l$ groter dan $0$. Dit leidt dus tot een contradictie.
\end{enumerate}
\end{answer}
\end{question}

\begin{question}[Operaties op (niet-)reguliere talen]
Bewijs (bijvoorbeeld aan de hand van een constructie van DFA of
reguliere expressie) of geef een tegenvoorbeeld voor volgende
beweringen:
\begin{enumerate}
\item Als de taal $L$ niet-regulier is (er bestaat geen DFA die de
taal kan beslissen), dan is $L^{\star}$ ook niet regulier.

\item 
Twee gegeven talen $L_1$ en $L_2$ kan je mengen als volgt:

$meng(L_1,L_2) = $

$~~~~~~~~~~~~\{s_1t_1s_2t_2\ldots s_nt_n | s\in L_1,t\in L_2, |s| = |t|,s = s_1s_2\ldots s_n, t = t_1t_2\ldots t_n, s_i \in \Sigma_1, t_i \in \Sigma_2\}$

Als $L_1$ en $L_2$ beide regulier zijn, dan is $meng(L_1,L_2)$ ook regulier.
\end{enumerate}
\begin{answer}
\begin{enumerate}
 \item Deze uitspraak is \textbf{fout}. Stel een taal $L=\condset{a^q}{q\in\PPP}$ met $\PPP$ de set van priemgetallen. $L$ is geen reguliere taal:
 \begin{quote}
 \begin{proof}
 We bewijzen dat $L\notin\mbox{RegLan}$ aan de hand van het pompend lemma. Neem een string $s=a^q$ met $q$ een priemgetal groter dan de onbekende pomplengte $p$. In dat geval bestaat er een opdeling $s=xyz$. Het is evident dat $y$ uit \'e\'en of meerdere $a$'s bestaat, m.a.w. $y=a^l$ met $l>0$. De lengte $\abs{xz}=q-l$. In het geval we $i$ gelijk nemen aan $i=q-l$ bekomen we de string $xy^{q-l}z=a^{l\cdot\brak{q-l}+q-l}=a^{\brak{q-l}\cdot\brak{l+1}}$. Het aantal $a$'s is hier dus deelbaar door $q-l$ en $l+1$, bijgevolg is $\brak{q-l}\cdot\brak{l+1}$ geen priemgetal, en behoort de string $xy^{q-l}z$ niet tot $L$.
 \end{proof}
 \end{quote}
 $L^{\star}$ is echter wel regulier:
 \begin{quote}
 \begin{proof}
 We weten dat zowel $2$ en $3$ in $\PPP$ zitten. We kunnen elk natuurlijk getal groter dan $1$ schrijven als een som van $2$ en $3$ per inductie:
 \begin{enumerate}
  \item $2$ is te schrijven als $2$ (een som met slechts \'e\'en element)
  \item Stel dat $n$ te schrijven valt als de som van elementen uit $\accl{2,3}$ dan geldt dit ook voor $n+1$. Indien de som voor $n$ immers een $2$ bevat, kunnen we de $2$ vervangen door een $3$; indien $n$ enkel $3$'en bevat, vervangen we een $3$ door $2+2$.
 \end{enumerate}
 Bijgevolg is $L^{\star}=\epsilon|aaa^{\star}$
 \end{proof}
 \end{quote}
 \item Deze uitspraak is \textbf{waar}. Hieronder geven we een constructie.
 \begin{quote}
 \begin{construction}[Meng-DFA]
 Vermits $L_1$ en $L_2$ regulier zijn bestaan er DFA's $\mbox{DFA}_1=\tupl{Q_1,\delta_1,\Sigma_1,F_1,q_{0,1}}$ en $\mbox{DFA}_2=\tupl{Q_2,\delta_2,\Sigma_2,F_2,q_{0,2}}$. De constructie resulteert in $\mbox{meng-DFA}=\tupl{Q,\delta,\Sigma,F,q_0}$ met:
 \begin{eqnarray}
  Q&=&\brak{Q_1\times Q_2\times\accl{0,1}}\cup\accl{q_w}\\
  \Sigma&=&\Sigma_1\cup\Sigma_2\\
  q_0&=&q_{0,1}\times q_{0,2}\times 0\\
  F&=&F_1\times F_2\times\accl{0}\\
  \forall a\in \Sigma:\fun{\delta}{q_w,a}&=&q_w\\
  \forall q_1\in Q_1,q_2\in Q_2,a\in \Sigma_1:\fun{\delta}{q_1\times q_2\times 0,a}&=&\fun{\delta_1}{q_1,a}\times q_2\times 1\\
  \forall q_1\in Q_1,q_2\in Q_2,a\in \Sigma_2:\fun{\delta}{q_1\times q_2\times 1,a}&=&q_1\times\fun{\delta_2}{q_2,a}\times 0\\
  \forall q_1\in Q_1,q_2\in Q_2,a\in \Sigma\setminus\Sigma_1:\fun{\delta}{q_1\times q_2\times 0,a}&=&q_w\\
  \forall q_1\in Q_1,q_2\in Q_2,a\in \Sigma\setminus\Sigma_2:\fun{\delta}{q_1\times q_2\times 1,a}&=&q_w
 \end{eqnarray}
 Met $q_w$ een nieuwe toestand zodat $q_w\notin\brak{Q_1\times Q_2\times\accl{0,1}}$.
 \end{construction}
 \end{quote}
 De intu\"itie is dat een toestand $q\in Q$ de toestanden van de twee originele DFA's bijhoudt alsook weet of het op dat moment verder moet in met de eerste of de tweede DFA. Indien het index-cijfer\footnote{Het derde component van een toestand.} op $0$ staat, is het de beurt aan de eerste DFA. Indien het ingelezen karakter tot $\Sigma_1$ behoort, passen we de eerste toestand aan volgens $\delta_1$ en zetten we het index-cijfer op $1$. Indien het karakter niet tot $\Sigma_1$ behoort, komen we in een reject-toestand $q_w$ terecht. Analoog construeren we voor de tweede DFA. De nieuwe DFA kan enkel accepteren als het index-cijfer op $0$ staat: de tweede DFA is dan even vaak aan de beurt geweest als de eerste.
\end{enumerate}
\end{answer}
\end{question}

\begin{question}[Contextvrije talen]
Geef voor onderstaande talen een context-vrije grammatica, een
push-down automaat, of bewijs aan de hand van het pompend lemma dat de
taal niet context-vrij is.
\begin{enumerate}
 \item $\condset{a^{2^n}}{n\in\NNN}$.
 \item $\condset{s\in\accl{\tfrac{0}{0},\tfrac{0}{1},\tfrac{1}{0},\tfrac{1}{1}}^{\star}}{\mbox{de binaire waarde bovenste rij is $3\times$ de binaire waarde onderste rij}}$.
 \item $\condset{ww^Rw}{w\in\accl{0,1}^{\star}}$. $w^R$ is de string $w$ omgekeerd.
 \item Een taal met $\Sigma=\accl{(,),[,]}$ volgens de normale regels
van haakjes, maar waarbij de vierkante haken op ieder niveau altijd na
de ronde haken komen te staan. Bijvoorbeeld $\brak{}\fbrk{}$,
$\brak{\fbrk{}}{\brak{}}$ en $\brak{\brak{}\fbrk{\brak{}}}\brak{}$ is
correct, maar $\fbrk{}\brak{}$ en
$\brak{\fbrk{}\brak{}\fbrk{}}\fbrk{}$ niet.
\end{enumerate}
\begin{answer}
\begin{enumerate}
 \item Dit is \textbf{niet context-vrij}. We bewijzen aan de hand van het pompend lemma. Stel $s$ een string $s=a^q$ met $q$ een macht van $2$ en $q>p$ met $p$ de onbekende pomplengte. Voor elke mogelijke opdeling $s=uvxyz$ weten we dat $v$ en $y$ uitsluitend uit $a$'s kunnen bestaan, we stellen $v=a^k$ en $y=a^l$. We weten dat moet gelden $0<l+k<p<q$. Stel $i=2$, dan is de gegenereerde string $a^{q+l+k}$. Vermits $q<q+k+l<2\cdot q$ is dit geen macht van $2$. Bijgevolg behoort deze string niet tot $L$.
 \item Deze taal is \textbf{regulier}. \figref{mrt-q5-dfa} toont hiervoor een DFA. De machine werkt \emph{big-endian}\footnote{Bij een big-endian machine worden eerst de minst significante bits ingevoerd.} en houdt telkens de overdracht bij. In het begin is de overdacht $0$. Wanneer er op de bovenste lijn een $1$ verschijnt, moet er onderaan $3$ worden toegevoegd. Vermits er slechts \'e\'en bit onderaan beschikbaar is, staat daarop een $1$ en schrijven we de rest (de overdacht) weg in een toestand. Omdat de overdacht maximum $4$ bereikt, kan deze taal worden beslist met een DFA. Indien we een \emph{little-endian} machine wensen te construeren kunnen we een constructie gebruiken om een DFA te bouwen die de omgekeerde taal beslist. Vermits elke DFA een PDA is waarbij de stapel niet wordt gebruikt, is de taal dus context-vrij.\\
 \importtikzfigure{mrt-q5-dfa}{Een DFA die binaire vermenigvuldiging met $3$ controleert.}
 Een interessant gevolg is dus dat men in digitale logica een \emph{online vermenigvuldiger} kan bouwen die in lineaire tijd voor een vaste constante een sequentieel ingevoerd binair getal, een binair getal kan genereren waarvan de invoer vermenigvuldigd is met de gegeven constante.
 \item Deze taal is \textbf{niet context-vrij}. Dit bewijzen we door eerst de doorsnede met $a^{\star}b^{\star}a^{\star}b^{\star}$ te nemen. In dat geval bekomen we de taal $L'=\condset{a^nb^{2\cdot n}a^{2\cdot n}b^n}{n\in\NNN}$. Een voor de hand liggende keuze voor $s$ is $s=a^pb^{2\cdot p}a^{2\cdot p}b^p$, met $p$ de onbekende pomplengte. Voor alle opdelingen $s=uvxyz$ weten we, op basis van de voorwaarden van het pompend lemma. Dat $v$ en $y$ elk uit slechts \'e\'en soort karakters kunnen bestaan. Onafhankelijk van welke karakters gekozen worden, kunnen hoogstens twee van de vier groepen verder worden uitgebreid. Bijgevolg is er geen enkele opdeling waarvoor we de taal kunnen pompen. Bijgevolg is de taal niet context-vrij.
 \item Deze taal is \textbf{context-vrij}. Zie \grmref{mrt-q5-grm} voor een context-vrije grammatica.
 \importgram{mrt-q5-grm}{Een contex-vrije grammatica voor de haakjes-taal.}
 De grammatica werkt als volgt: de eerste non-terminal $S$ wordt opgesplitst in non-terminals $A$ en $B$. $A$ zorgt voor ronde haken, $B$ voor vierkante haken. Op die manier voldoen we zeker aan de beperking dat vierkante haken na de ronde haken moeten komen. $A$ en $B$ kunnen zowel een lege string ($\epsilon$) genereren ofwel tussen de relevante haken $S$ en hierna opnieuw respectievelijk $A$ en $B$, deze staan toe dat we meerdere ronde/vierkante haken na elkaar zetten. Binnen de haken verwijzen we naar $S$ zodat we een cascade-effect kunnen bekomen).
\end{enumerate}
\end{answer}
\end{question}

\begin{question}[] De $\oplus$NFA\footnote{Uitspraak Parity-NFA}
\vspace{-0.5cm}
\begin{quote}
\begin{definition}
Een $\oplus$NFA is gedefinieerd door een tuple zoals een $NFA$, maar de
definitie van acceptatie door een $\oplus$NFA is verschillend: de
$\oplus$NFA accepteert een string $s$ wanneer $s$ door de NFA door een
oneven aantal toestanden in de $NFA$ geaccepteerd wordt, en verwerpt
anders.
\end{definition}
\end{quote}
Welke uitspraken zijn waar, beargumenteer of geef een tegenvoorbeeld:
\begin{enumerate}
\item
Elke DFA is een $\oplus$NFA en deze $\oplus$NFA bepaalt dezelfde taal.

\item
Een $\oplus$NFA zonder $\epsilon$-bogen en met voor elke toestand en
karakter juist \'{e}\'{e}n boog, is een DFA die dezelfde taal beslist.

\item
Elke taal bepaald door een $\oplus$NFA is context-vrij.

\item
Elke taal bepaald door een $\oplus$NFA is regulier.
\end{enumerate}
\begin{answer}
\begin{enumerate}
 \item \textbf{Waar}. Een DFA bevindt zich telkens in slechts \'e\'en toestand. Wanneer dit een accepterende toestand is, is het aantal accepterende toestanden waarin we ons bevinden oneven, dus zou de $\oplus\mbox{NFA}$ ook accepteren. Wanneer we ons niet in accepterende toestand bevinden, is het aantal accepterende toestanden even, zowel de DFA en de $\oplus\mbox{NFA}$ zouden dus verwerpen.
 \item \textbf{Waar}. Een NFA die aan deze voorwaarden voldoet is een DFA. Immers betekent dit dat $\delta$ een totale functie wordt met $\funsig{\delta}{Q\times\Sigma}{R}$ met $R=\powset{Q}$ zodat $\# R=1$. Bijgevolg kunnen we $R$ schrijven als $Q$ en wordt $\funsig{\delta}{Q\times\Sigma}{Q}$. Dit is de signatuur van $\delta$ voor een DFA.
 \item \textbf{Waar}. Volgend item toont aan dat een $\oplus\mbox{NFA}$ uitsluitend reguliere talen kan beslissen. Omdat elke reguliere taal ook context-vrij is, is deze uitspraak waar.
 \item \textbf{Waar}. We kunnen analoog aan een NFA, een constructie maken voor een $\oplus\mbox{NFA}$ zodat we een DFA bekomen die dezelfde taal accepteert.
 \begin{quote}
 \begin{construction}[$\oplus\mbox{NFA}$ naar DFA]
 Stel een $\oplus\mbox{NFA}=\tupl{Q,\Sigma,\delta,q_0,F}$, dan construeren we een $\mbox{DFA}=\tupl{Q',\Sigma,\delta',q_0',F'}$ als volgt:
 \begin{eqnarray}
  Q'&=&\powset{Q}\\
  q_0'&=&\fun{E}{q_0}\\
  \forall q'\in Q',a\in\Sigma:\fun{\delta'}{q',a}&=&\displaystyle\bigcup_{q\in q'}\fun{E}{\fun{\delta}{q,a}}\\
  F'&=&\condset{q'\in Q'}{\#\brak{q'\cap F}\mod 2=1}
 \end{eqnarray}
 Met $E$ gedefinieerd als in \S4.3.
 \end{construction}
 \end{quote}

\end{enumerate}

\end{answer}
\end{question}
\end{document}
